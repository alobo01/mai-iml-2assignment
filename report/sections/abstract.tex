We investigate the clustering performance of \textbf{K-Means}, \textbf{Global K-Means}, \textbf{X-Means}, \textbf{Fuzzy C-Means (FCM)}, and \textbf{Spectral Clustering} across three datasets: \textbf{Hepatitis}, \textbf{Mushroom}, and \textbf{Pen-based}. Each algorithm is assessed with various hyperparameters, distance metrics (\textbf{Euclidean}, \textbf{Manhattan}, \textbf{Clark}), and cluster validity indices (\textbf{ARI}, \textbf{NMI}, \textbf{DBI}, \textbf{Silhouette}, \textbf{CHS}). Dimensionality reduction techniques (\textbf{PCA}, \textbf{UMAP}) support visualization and interpretation. Our results show that \textbf{Global K-Means} improves consistency over standard \textbf{K-Means}, while \textbf{FCM} demonstrates flexibility with datasets of differing density and structure. These findings highlight the importance of algorithmic configuration and data characteristics in achieving robust clustering performance.