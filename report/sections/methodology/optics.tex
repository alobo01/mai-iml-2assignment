\subsection{Ordering Points To Identify the Clustering Structure (OPTICS)}

The OPTICS (Ordering Points To Identify the Clustering Structure) method is a density-based clustering algorithm designed to reveal
 the intrinsic structure of data without requiring explicit cluster assignments or fixed parameter settings. Instead of directly
  generating clusters, OPTICS produces an ordered list of data points annotated with metrics that reflect their density relationships, 
  such as core distances and reachability distances. This ordering captures the clustering structure across a range of density levels,
   allowing hierarchical and arbitrary-shaped clusters to be identified. 
% Visualization tools, like reachability plots, make it possible to interpret and extract clusters effectively, showcasing transitions 
% between dense regions and noise. 
The algorithm is computationally efficient, with performance similar to DBSCAN, and excels in uncovering the natural distribution of 
complex datasets.