\subsection{Ordering Points To Identify the Clustering Structure (OPTICS)}

The OPTICS (Ordering Points To Identify the Clustering Structure) method is a density-based clustering algorithm designed
to reveal the intrinsic structure of data without requiring explicit cluster assignments or fixed parameter settings.
Instead of directly generating clusters, OPTICS produces an ordered list of data points annotated with metrics that
reflect their density relationships, such as core distances and reachability distances. This ordering captures the
clustering structure across a range of density levels, allowing hierarchical and arbitrary-shaped clusters to be identified \cite{10.1145/304181.304187}.
For this work, the OPTICS method was implemented using the \textbf{sklearn} library, which provides a robust and efficient implementation for clustering tasks.

\subsubsection{Hyperparameters}
\begin{itemize}
    \item \textbf{Distance metric:} it determines the spatial relationships that underpin density-based
    clustering. Three different metrics were chosen for the three datasets: the \textbf{Euclidean} distance, \textbf{Manhattan}
     distance, and \textbf{Chebyshev} distance, and their definitions can be consulted in the Apendix section \ref{apendix}.
\end{itemize}

\begin{itemize}
    \item \textbf{Algorithm:} it specifies the method used to compute the nearest neighbors during the clustering process. The ones
    studied are detailed below.
    \begin{itemize}
        \item \textbf{Brute Force:} This method computes distances between all point pairs, ensuring compatibility with all
        datasets but with a computational complexity of $O(n^2)$.
        \item \textbf{Ball Tree:} A hierarchical structure is built, partitioning data into spherical regions.
        \item \textbf{KD Tree:} Similar to ball trees, KD trees partition data hierarchically but with axis-aligned splits.
    \end{itemize}
\end{itemize}

\begin{itemize}
    \item \textbf{xi}: This parameter influences the steepness of changes in the reachability plot used to
    delineate clusters. Smaller values allow for finer differentiation between clusters, whereas larger values merge
     clusters with more gradual density transitions.
    \item \textbf{min\_cluster\_size}: Sets the minimum number of points required for a group to be considered
     a cluster. This prevents very small, potentially noisy groups from being identified as clusters.
    \item \textbf{min\_samples}: Defines the minimum number of points required to qualify as a core point,
    impacting the density threshold for cluster membership. Higher values favor dense and well-defined clusters,
    excluding sparser areas.
\end{itemize}

These parameters work in unison to balance sensitivity to noise, cluster granularity, and robustness, enabling OPTICS
to adapt to diverse datasets and clustering tasks \cite{scikit-learn-optics}.
