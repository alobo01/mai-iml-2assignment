\subsection{Ordering Points To Identify the Clustering Structure (OPTICS)}

The OPTICS (Ordering Points To Identify the Clustering Structure) method is a density-based clustering algorithm designed
to reveal the intrinsic structure of data without requiring explicit cluster assignments or fixed parameter settings.
Instead of directly generating clusters, OPTICS produces an ordered list of data points annotated with metrics that
reflect their density relationships, such as core distances and reachability distances. This ordering captures the
clustering structure across a range of density levels, allowing hierarchical and arbitrary-shaped clusters to be identified.
The algorithm is computationally efficient, with performance similar to DBSCAN, and excels in uncovering the natural
distribution of complex datasets. For this work, the OPTICS method was implemented using the \textbf{sklearn} library, which provides a robust and efficient implementation for clustering tasks.

A key component of OPTICS is the choice of a distance metric, which defines how proximity between points is calculated.
This metric determines the spatial relationships that underpin density-based clustering. Three different metrics were chosen
for the three datasets: the Euclidean distance, Manhattan distance, and Chebyshev distance.

Moreover, three different methods were computed for the neighborhood queries, a fundamental operation in density-based
clustering:
\begin{itemize}
    \item \textbf{Brute Force:} This method computes distances between all point pairs, ensuring compatibility with all
    datasets but with a computational complexity of $O(n^2)$, making it less efficient for large datasets.
    \item \textbf{Ball Tree:} A hierarchical structure is built, partitioning data into spherical regions, which allows
    faster neighborhood searches for datasets with low to moderate dimensionality.
    \item \textbf{KD Tree:} Similar to ball trees, KD trees partition data hierarchically but with axis-aligned splits,
    making them particularly effective for moderately low-dimensional spaces.
\end{itemize}

Finally, several parameters in OPTICS were adjusted based on the dataset, reflecting the inherent variability between
datasets. These parameters show a crucial influence on cluster definition and extraction.
\begin{itemize}
    \item xi: This parameter influences the steepness of changes in the reachability plot used to
    delineate clusters. Smaller values allow for finer differentiation between clusters, whereas larger values merge
     clusters with more gradual density transitions.
    \item min\_cluster\_size: Sets the minimum number of points required for a group to be considered
     a cluster. This prevents very small, potentially noisy groups from being identified as clusters.
    \item min\_samples: Defines the minimum number of points required to qualify as a core point,
    impacting the density threshold for cluster membership. Higher values favor dense and well-defined clusters,
    excluding sparser areas.
\end{itemize}

These parameters work in unison to balance sensitivity to noise, cluster granularity, and robustness, enabling OPTICS
to adapt to diverse datasets and clustering tasks.
