\subsection{Spectral Clustering}
Spectral Clustering is a graph-based clustering technique that leverages the spectral properties of the similarity matrix
 to group data points. It begins by constructing a similarity graph from the input data, where nodes represent data points,
 and edges encode pairwise similarities based on a given affinity function. The graph Laplacian matrix, derived from the
 similarity graph, captures the structure of the data. By solving an eigenvalue problem on the Laplacian matrix, the
 algorithm embeds the data into a lower-dimensional space where traditional clustering methods can be applied to partition the
  data into distinct clusters \cite{spectral}. 


The \textit{SpectralClustering} algorithm was implemented from the \textbf{scikit-learn} library due to its robust and
 efficient approach to graph-based clustering.


 \subsubsection{Hyperparameters}
\begin{itemize}
  \item \textbf{assign\_labels:} is the algorithm used for clustering the data in the lower-dimensional space.
 \begin{itemize}
   \item \textbf{k-means}: This is the default method and groups points by minimizing the sum of squared distances within clusters. It is effective for well-separated clusters and computationally efficient for most applications.
   \item \textbf{cluster\_qr}: This method uses QR decomposition to cluster points, providing a numerically stable alternative to 
   \texttt{k-means}. It can be advantageous for datasets with unique cluster structures or when \textbf{k-means} struggles to
    converge.
\end{itemize}
\end{itemize}
\begin{itemize}
  \item \textbf{affinity:} This parameter defines the method used to compute the 
  similarity (or adjacency) matrix between data points, which was \textbf{nearest\_neighbors} 
  for all the cases, with different values for the parameter \textbf{n\_neighbors} depending on the size of the dataset.
\end{itemize}

\begin{itemize}
  \item \textbf{eigen\_solvers:} methods used to compute the eigenvectors and eigenvalues of the graph Laplacian,
  a key step in spectral clustering.
  \begin{itemize}
    \item \textbf{lobpcg}: it uses the Locally Optimal Block Preconditioned Conjugate Gradient method, which is efficient for 
    large-scale problems and works well with sparse matrices. 
    \item \textbf{amg}: the Algebraic Multigrid solver is particularly effective for problems with a well-structured 
    Laplacian matrix and leverages multilevel techniques for computational efficiency.
    \item \textbf{arpack}: it employs iterative methods to compute a few eigenvalues and eigenvectors, offering a
     balance between accuracy and computational cost.
\end{itemize}
  \item \textbf{eigen\_tol: } it controls the precision of solving the eigenvalue problem \cite{scikit-learn-spectralclustering}.
\end{itemize}



