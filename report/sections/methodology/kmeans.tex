\subsection{K-Means}\label{sec:kmeans}
K-means is a clustering algorithm used to group data into \(k\) clusters based on their features. It works by randomly 
initializing \(k\) centroids, assigning each data point to the nearest centroid using a distance metric, and then updating
 the centroids as the mean of the assigned points. This process repeats iteratively until the centroids stabilize or a set
  number of iterations is reached. The goal is to minimize the within-cluster variance, making the points in each cluster
   as similar as possible while keeping clusters distinct.

\subsubsection{Hyperparameters}
\begin{enumerate}
    \item \textbf{k:} The number of clusters to partition the dataset. Determines the complexity and granularity of the clustering.
    \item \textbf{Distance Metrics:} It measures the similarity between data points and cluster centroids, helping assign each point to the nearest 
    cluster. The distance metrics analyzed were the Euclidean distance, the Manhattan distance and the Clark distance, and
    their definitions can be consulted in the Apendix section \ref{apendix}.

    \item \textbf{Additional Parameters:}
    \begin{itemize}
        \item \textbf{Initial Centroids:} Pre-defined initial cluster centers used as the starting point for the clustering algorithm.
        \item \textbf{Maximum Iterations:} Limits the number of iterations to prevent excessive computational time, with a default of 10 iterations.
    \end{itemize}
\end{enumerate}

\subsubsection{Clustering Methodology}
\begin{itemize}
    \item \textbf{Clustering Process:}
    \begin{enumerate}
        \item Assign each data point to the nearest centroid using the specified distance metric.
        \item Recalculate centroids by computing the mean of all points in each cluster.
        \item Repeat assignment and recalculation until convergence or maximum iterations are reached.
    \end{enumerate}
        \item \textbf{Convergence Criteria:} 
    \begin{itemize}
        \item Clusters are considered stable when centroids no longer significantly change between iterations.
    \end{itemize}

    \item \textbf{Variance Computation:}
    \begin{itemize}
        \item Total within-cluster variance (E) is calculated by summing squared distances of points to their respective cluster centroids.
        \item Provides a measure of clustering compactness and quality.
    \end{itemize}
\end{itemize}
This methodology allows for flexible clustering configurations, enabling analysis across different datasets and hyperparameter values.
