\subsection{K-Means}\label{sec:kmeans}
K-Means intro.

\subsubsection{Hyperparameters}
\begin{enumerate}
    \item \textbf{k:}
    \begin{itemize}
        \item The number of clusters to partition the dataset. Determines the complexity and granularity of the clustering.
    \end{itemize}
    \item \textbf{Distance Metrics:}
    \begin{itemize}
        \item \textbf{Euclidean Distance:} Calculates the root of the sum of squared differences between feature values. Standard metric for continuous data:
        \[
        d(x, y) = \sqrt{\sum_{i=1}^{n} (x_i - y_i)^2}
        \]
        \item \textbf{Manhattan Distance:} Computes the sum of absolute differences between feature values, suitable for both categorical and continuous data:
        \[
        d(x, y) = \sum_{i=1}^{n} |x_i - y_i|
        \]
        \item \textbf{Clark Distance:} Accounts for proportional differences between feature values, enhancing interpretability for attributes with varying scales:
        \[
        d(x, y) = \sqrt{\sum_{i=1}^{n} \left(\frac{|x_i - y_i|}{x_i + y_i + \epsilon}\right)^2}
        \]
        where $\epsilon$ is a small constant to avoid division by zero.
    \end{itemize}

    \item \textbf{Additional Parameters:}
    \begin{itemize}
        \item \textbf{Initial Centroids:} Pre-defined initial cluster centers used as the starting point for the clustering algorithm.
        \item \textbf{Maximum Iterations:} Limits the number of iterations to prevent excessive computational time, with a default of 10 iterations.
    \end{itemize}
\end{enumerate}

\subsubsection{Clustering Methodology}
\begin{itemize}
    \item \textbf{Clustering Process:}
    \begin{enumerate}
        \item Assign each data point to the nearest centroid using the specified distance metric.
        \item Recalculate centroids by computing the mean of all points in each cluster.
        \item Repeat assignment and recalculation until convergence or maximum iterations are reached.
    \end{enumerate}
        \item \textbf{Convergence Criteria:} 
    \begin{itemize}
        \item Clusters are considered stable when centroids no longer significantly change between iterations.
    \end{itemize}

    \item \textbf{Variance Computation:}
    \begin{itemize}
        \item Total within-cluster variance (E) is calculated by summing squared distances of points to their respective cluster centroids.
        \item Provides a measure of clustering compactness and quality.
    \end{itemize}
\end{itemize}
This methodology allows for flexible clustering configurations, enabling analysis across different datasets and hyperparameter values.
