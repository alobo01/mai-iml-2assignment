\subsection{Dimensionality Reduction Techniques}
To enhance the visualization of results, two dimensionality reduction methods were applied: 
\begin{itemize}
    \item Principal Component Analysis (PCA) is a statistical technique used to simplify complex datasets by reducing
    their dimensionality while retaining most of the original variability in the data. It transforms the data into a new coordinate
     system where the axes, called principal components, are ordered by the amount of variance they capture from the original dataset. 
     The first principal component captures the maximum variance, followed by the second, and so on.
     \item Uniform Manifold Approximation and Projection (UMAP) is a dimensionality reduction technique 
     that simplifies high-dimensional data into lower dimensions while preserving its inherent structure. UMAP builds a graph 
     that reflects the relationships between data points in the original space, using principles from Riemannian geometry and 
     algebraic topology. It then optimizes a low-dimensional embedding to maintain these relationships as closely as possible. 
     UMAP is particularly effective for visualizing complex datasets and identifying clusters, as it excels at preserving both 
     local and global structures.
\end{itemize}
