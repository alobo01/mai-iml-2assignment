\section{Results}\label{sec:results}
To systematically evaluate the different configurations of each clustering algorithm, the following procedure is followed to extract results for each of the 3 data sets, in order to later perform an analysis those results:

\begin{enumerate}
    \item \textbf{Data Preparation}: The dataset is loaded and the data samples are separated from their labels into separate 2 sets. This way, we perform the clustering analysis in a completely non-supervised way, and we then utilize the labels to extract supervised metrics of the cultering results.

    \item \textbf{Parameter Configuration}: A comprehensive set of values for the algorithm's hyperparameters is defined. These combinations reflect various ways to tune the clustering algorithm.

    \item \textbf{Model Evaluation}: For each parameter combination, the clustering algorithm is applied on the unlabeled data and then evaluated with different metrics. This step yields the following metrics: Adjusted Rand Score (ARI), Normalized Mutual Information (NMI), Davies-Bouldin Index(DBI), Silhouette score, Calinski-Harabasz Score (CHS), and execution time. Together, these metrics (the first 2 supervised, and the rest non-supervised) measure the effectiveness and efficiency of the clustering.

    \item \textbf{Results Compilation}: The performance metrics for each parameter combination are recorded in a structured format. These results are saved as a dataset that summarizes the outcomes of all evaluations, forming a basis for analysis. We save as well the cluster labels of all samples for each clustering algorithm that we run, so we can recover the same clusters in the posterior analysis.

    \item \textbf{Results Analysis}: After results are compiled across all configurations, quantitative and qualitative analysis is performed to identify common trends among the different algorithm configurations for each of the datasets. Additionally, we extract the top performing configuration according to each of the 5 evaluation metrics, in order to study common patterns and visualize the resulting clusters. This analysis helps determine the most reliable and effective parameter settings for accurate and efficient clustering for each dataset.
\end{enumerate}
Due to the vast ammount of information that can be extracted from the results, we will only explicitly showcase the most clarifying plots and results that we have extracted. However, all of the information is available in the \texttt{plots\_and\_tables} folder for each of the datasets, in the \texttt{code} attached to this report.

\textit{Note:} Since all of the tested datasets have high dimensionality, we use Principal Component Analysis (PCA) to extract the 2 principal components in order to generate the clustering scatter plots of the datasets for visualization purposes.

\subsection{K-Means}
We have tested on each dataset 57 different configurations of the K-Means algorithm, by using the 3 different distance metrics with 19 different values of the k (from 2 to 20). For each of these configurations, we have run the K-Means 10 times, in order to account for the randomness of the centroid initialization. This results in a total of 570 runs of the K-Means algorithm for each dataset. From the evaluation metrics extracted for each of these runs, we study the effect of each of the 2 hyperparameters and achieve conclusions about them through statistical analysis.

\subsubsection{Preliminary Study}
Before starting with the more rigorous statistical analysis, let us first observe some preliminary patterns about the measured metrics and the effect of each hyperparameter on the clustering performance.

In Figure \ref{fig:metrics_corr} we summarize the relationship between the different metrics that were measured. It is a matrix plot where the lower triangle is a heatmap of the Pearson correlations between each pair of metrics, the diagonal elements are the histogram distributions of values of each metric, and the upper triangular has for each pair of metrics the plot of their values for all runs. It is interesting to observe that, while we would expect all of the metrics to agree on the identified trends, there are some cases where the opposite behaviour is displayed. An example of this is the negative correlation between E (total variance) and DBI (Davies-Bouldin Index): since the DBI measures cluster compactness, we would expect it to directly correlate to E; however, we observe that there is a negative correlation between them. This specific figure was extracted from the results on the Mushroom dataset, but the conclusions are the same for the others (the plots can be found in the \texttt{code} floder).

\begin{figure}[h]
    \centering
    \includegraphics[width=0.7\textwidth]{figures/metrics_correlations_matrix.png}
    \caption{Metrics correlations summary}
    \label{fig:metrics_corr}
\end{figure}

Parallelly, a different set of interesting relationship are displayed in Figure \ref{fig:pairplot}, where we can see heatmaps of the F1 Score and the Time across the different pairwise hyperparameter configurations. We can observe a general trend regarding execution time: it seems to have considerably larger values for the Clark distance metric compared to those of the other 2, which reflects the higher computational cost that this distance metric has. Additionally, we see a noteworthy divergence in the F1 Score trends with respect to the values of k: in the Mushroom dataset (which has 2 classes), lower values of k seem to achieve a better F1 Score; meanwhile, in the Pen-based dataset (which has 10 classes), intermediate values of k (between 7 and 11) seem to achieve the best scores. This was to be expected, yet it still is compelling to see it reflected so clearly in the results. 

\begin{figure}[H]
    \centering
    \begin{subfigure}{0.49\textwidth}
        \centering
        \includegraphics[width=\linewidth]{figures/mushroom_hyperparameter_pairplot_matrix.png}
        \caption{Mushroom pairplot matrix}
    \end{subfigure}
    \hfill
    \begin{subfigure}{0.49\textwidth}
        \centering
        \includegraphics[width=\linewidth]{figures/penbased_hyperparameter_pairplot_matrix.png}
        \caption{Pen-based pairplot matrix}
    \end{subfigure}
    \caption{Hyperparameter pairplot matrices based on F1 Score and Time}
    \label{fig:pairplot}
\end{figure}

\subsubsection{Statistical Analysis}
For the statistical analysis, we have first performed Friedman tests to determine if there are significant differences in each of the metrics for the different possible values of each hyperparameter. 


\subsection{Global K-Means}
We have tested on each dataset 171 different configurations of the Global K-Means algorithm, by using the 3 different distance metrics with 19 different values of the k (from 2 to 20), and 3 different values for the number of buckets (2k, 3k and 4k). This time, we have only run the Global K-Means once for each of these configurations, since it is a deterministic algorithm that would return the same results every time. From the evaluation metrics extracted for each of these runs, we study the effect of each of the 3 hyperparameters and study the best performing runs according to each of the metrics.

\subsubsection{Hyperparameter Study}
As in the previous section, we start by observing some preliminary patterns about the effect of each hyperparameter on the clustering performance.

In Figure \ref{fig:globalkmeans:violin} we summarize the observed trends in each of the datasets (columns) for each of the hyperparameters (rows) with different metrics. It is not surprising to see that the patterns regarding the K-Means hyperparameters (k and distance metric) mostly repeat for the Global K-Means, since Global K-Means is mearly a modification of the same idea as the standard K-Means.

\begin{itemize}
    \item In the first row of plots, we observe that Hepatitis again favors low values of k; Mushroom does not achieve very conclusive results, but definitely gets consistently worse with higher values of k; and Pen-based benefits from intermediate values, getting more inconsistent for the highest ones.
    \item In the middle row, it is evident that the Clark distance performs significantly better on the Hepatitis dataset and significantly worse on the other 2 datasets, which are similar conclusions as we had reached before.
    \item In the last row, we observe that the clustering performance generally does not seem to be significantly altered by the different values of the number of buckets that we use to initilize the candidate points of the algorithm.
\end{itemize}


\begin{figure}[h]
    \centering
    \begin{tabular}{ccc}
        \includegraphics[width=0.3\textwidth]{figures/GlobalKMeans/hepatitis_violin_k_vs_CHS.png} & 
        \includegraphics[width=0.3\textwidth]{figures/GlobalKMeans/mushroom_violin_k_vs_CHS.png} & 
        \includegraphics[width=0.3\textwidth]{figures/GlobalKMeans/penbased_violin_k_vs_NMI.png} \\
        \includegraphics[width=0.3\textwidth]{figures/GlobalKMeans/hepatitis_violin_Distance_Metric_vs_ARI.png} & 
        \includegraphics[width=0.3\textwidth]{figures/GlobalKMeans/mushroom_violin_Distance_Metric_vs_Silhouette.png} & 
        \includegraphics[width=0.3\textwidth]{figures/GlobalKMeans/penbased_violin_Distance_Metric_vs_Silhouette.png} \\
        \includegraphics[width=0.3\textwidth]{figures/GlobalKMeans/hepatitis_violin_N_Buckets_vs_CHS.png} & 
        \includegraphics[width=0.3\textwidth]{figures/GlobalKMeans/mushroom_violin_N_Buckets_vs_Silhouette.png} & 
        \includegraphics[width=0.3\textwidth]{figures/GlobalKMeans/penbased_violin_N_Buckets_vs_DBI.png} \\
    \end{tabular}
    \caption{Violin plots of the Global K-Means: Each row is a hyperparameter (k, distance metric, number of buckets), and each column is a dataset (Hepatitis, Mushroom, Pen-based)}
    \label{fig:globalkmeans:violin}
\end{figure}

\subsubsection{Best Runs}
As for the K-Means, we extracted for each of the datasets the run which achieved the best score for each of the metrics. A summary of them is displayed in Table \ref{tab:kmeans:best_runs}.

\textcolor{red}{TO DO}

\begin{table}[h!]
    \centering
    \begin{tabular}{|c|ccc|ccc|ccc|}
        \hline
                        & \multicolumn{3}{c|}{\textbf{Hepatitis}} & \multicolumn{3}{c|}{\textbf{Mushroom}} & \multicolumn{3}{c|}{\textbf{Pen-based}} \\ \hline
        \textbf{Metric} & \textbf{k} & \textbf{Distance} & \textbf{Value} 
                        & \textbf{k} & \textbf{Distance} & \textbf{Value} 
                        & \textbf{k} & \textbf{Distance} & \textbf{Value} \\ \hline
        ARI            & 2          & manhattan         & 0.37 
                       & 2          & clark         & 0.40 
                       & 14          & euclidean             & 0.64 \\ \hline
        NMI            & 3          & clark             & 0.25 
                       & 15          & clark         & 0.43 
                       & 14          & euclidean         & 0.74 \\ \hline
        DBI            & 20         & euclidean         & 1.40 
                       & 2         & euclidean             & 1.20 
                       & 7         & euclidean         & 1.23 \\ \hline
        Silhouette     & 2          & manhattan         & 0.21 
                       & 2          & euclidean         & 0.28 
                       & 10          & euclidean             & 0.32 \\ \hline
        CHS            & 2          & euclidean         & 36.77 
                       & 2          & euclidean         & 2996.24 
                       & 4          & euclidean         & 3361.02 \\ \hline
    \end{tabular}
    \caption{Metrics with corresponding values, k, and distance metrics for three datasets.}
    \label{tab:kmeans:best_runs}
\end{table}

From these results, we can make some deductions as to which are the best hyperparameter configurations of the K-Means for the 3 datasets:
\begin{enumerate}
    \item \textbf{Hepatitis:} It is again clear that low values of k (2 and 3) typically achieve better scores, which we had already observed before. On the other hand, it is not so clear which of the 3 distance metrics is more appropriate for this dataset, since all of them appear in the top scoring runs.
    \item \textbf{Mushroom:} We can observe once again that k=2 is dominant, probably due to the 2 classes that the dataset has. As for the distance metrics, Euclidean seems to be the most effective, followed by Clark, and Manhattan does not appear to be useful for the properties of this dataset.
    \item \textbf{Pen-based:} In this case, we do not see such a clear predominance of any specific value of k, but there seems to be a trend towards intermediate values, which was to be expected due to the 10 classes of the dataset. In contrast, we do see a constant primacy of the Euclidean distance metric over the rest, that clearly appears to be better at capturing key differences in this dataset than the other two.
\end{enumerate}
Additionally, we can observe that the 2 bigger datasets (Mushroom and Pen-based) have overall better top values of the metrics than the smaller, Hepatitis dataset. In particular, the Pen-based has the best results in all of the metrics except DBI (in which the difference is minimal), which leads us to the conclusion that it is the dataset for which the K-Means clustering algorithm is better suited.


% 
\subsection{Fuzzy C-Means}

We have tested 432 different configurations of the Fuzzy C-Means (FCM) algorithm on each dataset by using the fuzziness parameter \( m \) (with values from 1.5 to 4.0) and varying the number of clusters ( n\_clusters) between the following values:

\begin{itemize}
  \item Pen-based: 6, 8, 9, 10
  \item Mushroom and Hepatitis: 2, 3, 4, 5
\end{itemize}


The following parameters were also varied:


\begin{itemize}
  \item \( \text{max\_iter} \): 100, 300, 500
  \item \( \text{error} \): 1e-1, 1e-4, 1e-5
  \item \( \rho \): 0.5, 0.7, 0.9
\end{itemize}


For each configuration, we ran the algorithm 10 times to mitigate the effects of initialization randomness, resulting in a total of 4320 runs of the FCM algorithm per dataset. From the evaluation metrics extracted in these runs, we analyze the impact of the key hyperparameters and derive insights through statistical analysis.

We decided not to include \textit{max\_iter} and \textit{error tolerance} in the analysis of performance metrics, as they do not significantly affect the performance, aside from removing outliers and slightly improving execution time. To illustrate this, we include 3 heatmaps displaying execution times for each dataset, along with the distribution of ARI results for different values of \textit{max\_iter} and \textit{error tolerance} (see \textbf{Figure}\ref{fig:error-iter-fuzzy}).


\begin{figure}[H]
    \centering
    \begin{subfigure}{0.32\textwidth}
        \centering
        \includegraphics[width=\linewidth]{figures/FuzzyCMeans/penBased_error_iter_pairplot.png}
        \caption{Pen-based dataset.}
    \end{subfigure}
    \begin{subfigure}{0.32\textwidth}
        \centering
        \includegraphics[width=\linewidth]{figures/FuzzyCMeans/mushroom_error_iter_pairplot.png}
        \caption{Mushroom dataset.}
    \end{subfigure}
    \begin{subfigure}{0.32\textwidth}
        \centering
        \includegraphics[width=\linewidth]{figures/FuzzyCMeans/hepatitis_error_iter_pairplot}
        \caption{Hepatitis dataset.}
    \end{subfigure}
    \caption{Heatmaps illustrating execution times for each dataset, showcasing performance across different configurations.}
    \label{fig:error-iter-fuzzy}
\end{figure}


\subsubsection{Preliminary Study}

We first explored preliminary patterns in the measured metrics and the influence of hyperparameters on clustering performance.


\textbf{Figure} \ref{fig:metrics_corr_fuzzy} illustrates the relationships between the various metrics for the FCM algorithm. This matrix plot highlights the correlations between metrics (lower triangle), histogram distributions (diagonal), and scatterplots of metric values (upper triangle). A notable observation is the interaction between the fuzziness parameter and metrics like Silhouette and DBI. For instance, while lower fuzziness (mm) often improves silhouette scores due to sharper cluster boundaries, it can lead to higher DBI values, which might indicate a trade-off between cluster compactness and interpretability. Similar trends are observed across all datasets, though the degree of correlation varies.


\begin{figure}[H]
	\centering
	\begin{subfigure}{0.32\textwidth}
		\centering
		\includegraphics[width=\linewidth]{figures/FuzzyCMeans/penBased_metrics_correlations_matrix.png}
		\caption{Pen-based dataset.}
	\end{subfigure}
	\begin{subfigure}{0.32\textwidth}
		\centering
		\includegraphics[width=\linewidth]{figures/FuzzyCMeans/mushroom_metrics_correlations_matrix.png}
		\caption{Mushroom dataset.}
	\end{subfigure}
	\begin{subfigure}{0.32\textwidth}
		\centering
		\includegraphics[width=\linewidth]{figures/FuzzyCMeans/hepatitis_metrics_correlations_matrix.png}
		\caption{Hepatitis dataset.}
	\end{subfigure}
	\caption{Metrics correlation acrross the three datasets.}
	\label{fig:metrics_corr_fuzzy}
\end{figure}


Additionally, Figure \ref{fig:pairplot_fuzzy} shows hyperparameter pairwise relationships, particularly the impact of fuzziness $m$ and number of clusters n\_clusters on clustering quality and execution time. Across datasets, higher fuzziness values consistently resulted in more computationally expensive runs, likely due to the increased complexity of assigning membership values. Interestingly, for datasets like Pen-based (10 classes), high cluster counts (e.g., n\_clusters$=10$) aligned closely with ground truth and yielded better metrics, such as ARI and NMI. Conversely, for Mushroom (2 classes), smaller values of n\_clusters and lower fuzziness performed better.


\begin{figure}[H]
\centering
\begin{subfigure}{0.49\textwidth}
\centering
\includegraphics[width=\linewidth]{figures/FuzzyCMeans/mushroom_hyperparameter_pairplot_matrix.png}
\caption{Mushroom pairplot matrix for FCM}
\end{subfigure}
\hfill
\begin{subfigure}{0.49\textwidth}
\centering
\includegraphics[width=\linewidth]{figures/FuzzyCMeans/penBased_hyperparameter_pairplot_matrix.png}
\caption{Pen-based pairplot matrix for FCM}
\end{subfigure}
\caption{Hyperparameter pairplot matrices based on clustering metrics and execution time for FCM}
\label{fig:pairplot_fuzzy}
\end{figure}



\subsubsection{Fuzziness Parameter (m):}

\begin{figure}[H]
	\centering
	\begin{subfigure}{0.32\textwidth}
		\centering
		\includegraphics[width=\linewidth]{figures/FuzzyCMeans/penBased_violin_fuzziness_vs_DBI}
		\caption{Pen-based dataset.}
	\end{subfigure}
	\begin{subfigure}{0.32\textwidth}
		\centering
		\includegraphics[width=\linewidth]{figures/FuzzyCMeans/mushroom_violin_fuzziness_vs_Silhouette}
		\caption{Mushroom dataset.}
	\end{subfigure}
	\begin{subfigure}{0.32\textwidth}
		\centering
		\includegraphics[width=\linewidth]{figures/FuzzyCMeans/hepatitis_violin_fuzziness_vs_NMI}
		\caption{Hepatitis dataset.}
	\end{subfigure}
	\caption{Metrics correlation acrross the three datasets.}
	\label{fig:metrics_corr_fuzzy}
\end{figure}



\begin{comment}
\subsubsection*{Number of Clusters (n\_clusters):}

\begin{enumerate}

\item Hepatitis: Lower cluster counts (e.g., n_clusters=2n\_clusters = 2) showed better ARI and Silhouette scores, aligning with the dataset's two-class structure.

\item Mushroom: Metrics like CHS favored smaller cluster counts, with n_clusters∈[2,4]n\_clusters \in [2, 4] achieving the best results. Silhouette scores corroborated these findings.

\item Pen-based: Higher cluster counts (n_clusters=10n\_clusters = 10) aligned with the dataset's class structure and yielded optimal ARI and NMI. Lower counts led to poor clustering quality, while overly high values showed diminishing returns.

\end{enumerate}

These findings confirm the sensitivity of FCM to both hyperparameters and its suitability for datasets with well-defined class structures.



\subsection*{Highlights from Tables}

\begin{itemize}

\item Pen-based Dataset: Optimal ARI (0.6242) and NMI (0.7073) achieved with n_clusters=10n\_clusters = 10 and m=2.0m = 2.0, reflecting its 10-class nature.

\item Mushroom Dataset: CHS (1962.91) supports n_clusters=6n\_clusters = 6 and m=1.5m = 1.5, showcasing the importance of small cluster counts in this two-class dataset.

\item Hepatitis Dataset: Lower n_clustersn\_clusters and fuzziness values provide better interpretability, with the highest ARI (0.1389) for m=1.5m = 1.5.

\end{itemize}

This adaptation integrates Fuzzy C-Means characteristics while maintaining your original narrative structure.
\end{comment}

